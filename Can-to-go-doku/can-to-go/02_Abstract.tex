\addsec{Zusammenfassung / Abstract}
\label{sec:zusammenfassung}
\noindent Das Projekt "CAN-TO-GO-SYSTEM", entwickelt von Michael Graml und Leonard Kreil, stellt eine fortschrittliche Lösung für die Überwachung und Diagnose von CAN-Bussystemen dar. Es zielt darauf ab, Anwendern bei der Einrichtung und Fehlersuche in CAN-Netzwerken zu assistieren. Das System zeichnet sich durch eine Kombination von Hardware- und Softwarekomponenten aus, die eine effiziente Analyse und Interaktion mit dem CAN-Netzwerk ermöglichen.\\

\noindent Die Hardware besteht aus einem sorgfältig Leiterplattenlayout, das einen ESP32-Mikrocontroller, ein Display, verschiedene Anschlüsse und LEDs integriert. Der ESP32 spielt eine zentrale Rolle bei der Verarbeitung der CAN-Nachrichten und der Interaktion mit anderen Hardwarekomponenten. Das System bietet außerdem eine Benutzeroberfläche über ein kleines Display und eine ergänzende App sowie Web-Integration für eine detailliertere Ansicht und Analyse der CAN-Nachrichten.\\

\noindent Die Softwarearchitektur basiert auf dem ESP32, der eine zentrale Rolle bei der Verarbeitung und Darstellung der CAN-Nachrichten übernimmt. Zusätzlich wird eine benutzerfreundliche GUI über das Flutter-Framework bereitgestellt, die eine flexible Interaktion auf verschiedenen Geräten ermöglicht. Die Implementierung der Software folgt der Clean Architecture und den SOLID-Prinzipien, um eine effiziente und erweiterbare Lösung zu gewährleisten.\\

\newpage
\minisec{Abstract}
\label{abstract}

\noindent The "CAN-TO-GO-SYSTEM" project, developed by Michael Graml and Leonard Kreil, represents an advanced solution for monitoring and diagnosing CAN bus systems. It aims to assist users in setting up and troubleshooting CAN networks. The system is characterized by a combination of hardware and software components that enable efficient analysis and interaction with the CAN network.\\

\noindent The hardware consists of a meticulously designed PCB layout, incorporating an ESP32 microcontroller, a display, various connectors, and LEDs. The ESP32 plays a central role in processing CAN messages and interacting with other hardware components. Additionally, the system provides a user interface through a small display and complementary app and web integration for more detailed viewing and analysis of CAN messages.\\

\noindent The software architecture is based on the ESP32, which takes a central role in processing and displaying CAN messages. A user-friendly GUI is also provided through the Flutter framework, allowing flexible interaction across different devices. The software implementation follows Clean Architecture and SOLID principles to ensure an efficient and expandable solution.\\